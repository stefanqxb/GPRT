\documentclass{bioinfo}
\copyrightyear{2005}
\pubyear{2005}

\begin{document}
\firstpage{1}

\title[short Title]{Confidence Estimation for Peptide Retention-Time Prediction}
\author[Sample \textit{et~al}]{Corresponding Author\,$^{1,*}$, Co-Author\,$^{2}$ and Co-Author\,$^2$\footnote{to whom correspondence should be addressed}}
\address{$^{1}$Department of XXXXXXX, Address XXXX etc.\\
$^{2}$Department of XXXXXXXX, Address XXXX etc.}

\history{Received on XXXXX; revised on XXXXX; accepted on XXXXX}

\editor{Associate Editor: XXXXXXX}

\maketitle

\begin{abstract}

\section{Motivation:}
Text Text Text  Text Text Text Text Text Text Text Text
Text  Text Text Text Text Text Text Text Text Text  Text Text Text Text Text Text Text Text Text  Text Text Text Text Text Text Text Text Text  Text Text Text Text Text Text Text Text Text  Text Text Text Text Text Text Text Text Text  Text Text Text Text Text.

\section{Results:}
Text  Text Text Text Text Text Text Text Text Text  Text Text Text Text Text Text Text Text Text  Text Text Text Text Text Text Text Text Text  Text Text Text Text Text Text

\section{Availability:}
Text  Text Text Text Text Text Text Text Text Text  Text Text Text Text Text Text Text Text Text  Text Text Text Text Text Text Text Text Text  Text

\section{Contact:} \href{name@bio.com}{name@bio.com}
\end{abstract}

\section{Introduction}
\label{sec::intro}
The task of retention time prediction, focuses on determining the retention time of a peptide given its amino acid sequence. Accurate peptide retention-time prediction in protein mass spectrometry can highly increase the efficiency of peptides spectrum matches in data-independent acquisition (DIA). Similar to many other machine learning methods, this prediction is done by training a model on a training set and using it to evaluate the unseen peptides. In previous studies, different models such as artificial neural networks (ANN) \cite{Bishop:2006ui} and support vector regression (SVR) \cite{Bishop:2006ui} have been applied to this task. These models have been originally designed to solve classifications tasks and they show certain limitations when they are applied to regression tasks.

When using machine learning to predict an entity, we are always interested in the confidence of this prediction. Ideally, this confidence to reflect certain properties of the observed data with respect to the trained model. As an example, this property can reflect how close is the observation to samples that the model is trained on. Such a confidence measure can be very useful at the evaluation stage, due to the fact that most machine learning methods can only generalize near the data points they have observed during the training.

\section{Approach}
In this paper, our aim is to demonstrate how obtaining the confidence for retention time prediction can provide us with more accurate predictions for a large fraction of the test data. To conduct our experiments, we are following the methodology of \cite{elude}, in which elude features are used for describing peptide and mapping them into a vector space and support vector regression (SVR) is used for predicting the retention time prediction. Since SVR framework does not directly provide us with the confidence of its predictions, we either have to use methods such as \cite{DeBrabanter:2011he} to approximate the confidence interval of predictions, or use other learning frameworks such as Gaussian Process (GP) \cite{Rasmussen:2006vza} that natively provide us with confidence estimation. In this work, we will focus on the later approach and replace SVR with GP. 

We first focus on providing a theoretical background on how GP estimates the confidences of the prediction and later focus on the experiments conducted in this paper. The role of these experiments are to both benchmark the performance of GP vs SVR as it was used in \cite{elude} and analyze how we can improve our estimation using the confidence of the predictions.

This paper is organized as following : In \S\ref{sec::method}, we will describe the GP framework and describe the setting and methodology our experiments. In \S\ref{sec::results}, we present the outcome of our experiments. Finally, in \S\ref{sec::dis} and \S\ref{sec::con}, we discuss the properties of our framework and conclude the paper.
\begin{methods}
\section{Methods}
\label{sec::method}


\section{Results}
\label{sec::results}

- GPs are performing better than epsilon-SVRs (and RVMs)
* Plot of deltaRT95 as a function of training set size

-The Elude features give better performance (, but take longer time to calculate)
* Compare with Elude-RBF, BOW-RBF (1,2 and 3-mers), and spectrum kernel
* Plot of deltaRT95 as a function of training set size

- GPs are predicting confidence
* Plot a actual obs-pred RT error as a function of stdv.

- Our Method ... outperforms the state-of-the-art, (Elude and SSRC) 
* Plot observed vs predicted RT in colormaps in 3 different subplots
* Alternatively (and probably better), plot 3 histograms of obs-pred RT for the three methods, and overlay them using transparent colors (e.g. set their alpha-value)

- Our method can predict RT of PTMs
* Plot histogram of error (obs-pred RT) for a set of phospho peptides. 

- GP confidence, observed vs predicted
* Compare the test and train set

- GP confidence, Find a list of peptides and sort them according to their similarity to the training set
* Plot their confidence 



\begin{table}[!t]
\processtable{This is table caption\label{Tab:01}}
{\begin{tabular}{llll}\toprule
head1 & head2 & head3 & head4\\\midrule
row1 & row1 & row1 & row1\\
row2 & row2 & row2 & row2\\
row3 & row3 & row3 & row3\\
row4 & row4 & row4 & row4\\\botrule
\end{tabular}}{This is a footnote}
\end{table}

\end{methods}

\begin{figure}[!tpb]%figure1
%\centerline{\includegraphics{fig01.eps}}
\caption{Caption, caption.}\label{fig:01}
\end{figure}

\begin{figure}[!tpb]%figure2
%\centerline{\includegraphics{fig02.eps}}
\caption{Caption, caption.}\label{fig:02}
\end{figure}

\section{Discussion}
\label{sec::dis}










%%%%%%%%%%%%%%%%%%%%%%%%%%%%%%%%%%%%%%%%%%%%%%%%%%%%%%%%%%%%%%%%%%%%%%%%%%%%%%%%%%%%%
%
%     please remove the " % " symbol from \centerline{\includegraphics{fig01.eps}}
%     as it may ignore the figures.
%
%%%%%%%%%%%%%%%%%%%%%%%%%%%%%%%%%%%%%%%%%%%%%%%%%%%%%%%%%%%%%%%%%%%%%%%%%%%%%%%%%%%%%%






\section{Conclusion}
\label{sec::con}



\begin{enumerate}
\item this is item, use enumerate
\item this is item, use enumerate
\item this is item, use enumerate
\end{enumerate}

Text Text Text Text Text Text  Text Text Text Text Text Text Text Text Text  Text Text Text Text Text Text. Figure \ref{fig:02} shows that the above method  Text Text Text Text  Text Text Text Text Text Text  Text Text.  \citealp{Boffelli03} might want to know about  text text text text
Text Text Text Text Text Text  Text Text Text Text Text Text Text Text Text  Text Text Text Text Text Text. Figure \ref{fig:02} shows that the above method  Text Text Text Text  Text Text Text Text Text Text  Text Text.  \citealp{Boffelli03} might want to know about  text text text text
Text Text Text Text Text Text  Text Text Text Text Text Text Text Text Text  Text Text Text Text Text Text.






Text Text Text Text Text Text  Text Text Text Text Text Text Text Text Text  Text Text Text Text Text Text. Figure \ref{fig:02} shows that the above method  Text Text Text Text


\section*{Acknowledgement}
Text Text Text Text Text Text  Text Text.  \citealp{Boffelli03} might want to know about  text text text text

\paragraph{Funding\textcolon} Text Text Text Text Text Text  Text Text.

%\bibliographystyle{natbib}
%\bibliographystyle{achemnat}
%\bibliographystyle{plainnat}
%\bibliographystyle{abbrv}
%\bibliographystyle{bioinformatics}
%
%\bibliographystyle{plain}
%
%\bibliography{Document}


\begin{thebibliography}{}
\bibitem[Bofelli {\it et~al}., 2000]{Boffelli03} Bofelli,F., Name2, Name3 (2003) Article title, {\it Journal Name}, {\bf 199}, 133-154.

\bibitem[Bag {\it et~al}., 2001]{Bag01} Bag,M., Name2, Name3 (2001) Article title, {\it Journal Name}, {\bf 99}, 33-54.

\bibitem[Yoo \textit{et~al}., 2003]{Yoo03}
Yoo,M.S. \textit{et~al}. (2003) Oxidative stress regulated genes
in nigral dopaminergic neurnol cell: correlation with the known
pathology in Parkinson's disease. \textit{Brain Res. Mol. Brain
Res.}, \textbf{110}(Suppl. 1), 76--84.

\bibitem[Lehmann, 1986]{Leh86}
Lehmann,E.L. (1986) Chapter title. \textit{Book Title}. Vol.~1, 2nd edn. Springer-Verlag, New York.

\bibitem[Crenshaw and Jones, 2003]{Cre03}
Crenshaw, B.,III, and Jones, W.B.,Jr (2003) The future of clinical
cancer management: one tumor, one chip. \textit{Bioinformatics},
doi:10.1093/bioinformatics/btn000.

\bibitem[Auhtor \textit{et~al}. (2000)]{Aut00}
Auhtor,A.B. \textit{et~al}. (2000) Chapter title. In Smith, A.C.
(ed.), \textit{Book Title}, 2nd edn. Publisher, Location, Vol. 1, pp.
???--???.

\bibitem[Bardet, 1920]{Bar20}
Bardet, G. (1920) Sur un syndrome d'obesite infantile avec
polydactylie et retinite pigmentaire (contribution a l'etude des
formes cliniques de l'obesite hypophysaire). PhD Thesis, name of
institution, Paris, France.

\end{thebibliography}
\end{document}
